\documentclass[16pt]{article}

\usepackage{report}

\usepackage[utf8]{inputenc} % allow utf-8 input
\usepackage[english, russian]{babel}
\usepackage[T1]{fontenc}    % use 8-bit T1 fonts
\usepackage[colorlinks=true, linkcolor=black, citecolor=blue, urlcolor=blue]{hyperref}       % hyperlinks
\usepackage{url}            
\usepackage{booktabs}       
\usepackage{amsfonts}       
\usepackage{nicefrac}      
\usepackage{microtype} 
\usepackage{graphicx}
\usepackage{natbib}
\usepackage{mathrsfs}
\usepackage{doi}
\usepackage{mathtools}
\usepackage{graphicx}
\usepackage{listings}
\usepackage{pythonhighlight}
\usepackage{mathtools}
\graphicspath{ {./figs/} }


\setcitestyle{aysep={,}}



\title{ЛР \textnumero 4 <<Астатизмы>>}

\author{
Студент \\
Кирилл Лалаянц\\
R33352\\
336700\\
Вариант - 6\\
\\
Преподаватель\\
Пашенко А.В. \\
\\
\\
Факультет Систем Управления и Робототехники\\
\\
ИТМО\\
}

% Uncomment to remove the date
\date{13.09.2023}

% Uncomment to override  the `A preprint' in the header
\renewcommand{\headeright}{ЛР \textnumero 4 <<Астатизмы>>}
\renewcommand{\undertitle}{Отчет}
\renewcommand{\shorttitle}{}


\begin{document}
\maketitle
\newpage
\tableofcontents
\thispagestyle{empty}

\newpage
\setcounter{page}{1}
\section{Вводные данные}
\subsection{Цель работы}
В этой работе будет проведенно исследование следующих вопросов:
\begin{itemize}
    \item Астатизмы.
    \item Принцип внутренней модели.
    \item Идеальное и реальное дифференцирующие звенья
\end{itemize} 

\newpage
\section{Выполнение работы}
\label{sec:headings}


\subsection{Задание 1. Задача стабилизации с идеальным дифференцирующим звеном.}

\subsubsection{Теория}
В этом задании будет проведена симуляция системы с ПД регулятором, используя дифференциальное звено, для open- и closed-loop систем.
\subsubsection{Результаты}
Ожидаемо, замкнутая система успешно свела ошибку к 0, а открытая -- нет.
\begin{figure}[h!]
    \centering
    \includegraphics[width=\textwidth]{task1}
    \caption{Результат выполнения первого задания.}
    \label{fig:fig1}
\end{figure}

\pagebreak

\subsection{Задание 2. Задача стабилизации с реальным дифференцирующим звеном.}

\subsubsection{Теория}
В этом задании будет проведена симуляция системы с ПД регулятором, используя реальное дифференциальное звено.
Так же исследован параметр Т на предмет устойчивости.
\subsubsection{Результаты}
Заметно, что параметр \(Т=10^{-3}\) уже достаточно мал, и отличий с \(Т=10^{-5}\) нет. Граница устойчивости была получена через решение системы уравнений следующих из матрицы Гурвица. Ее значение экспериментально подтвердилось (было взято число чуть больше, поэтому график расходится).
\begin{figure}[h!]
    \centering
    \includegraphics[width=\textwidth]{task2}
    \caption{Результат выполнения второго задания.}
    \label{fig:fig2}
\end{figure}

\pagebreak

\subsection{Задание 3. Исследование влияния шума.}

\subsubsection{Теория}
В этом задании будет проведено исследование влияния шума на конечный результат.
\subsubsection{Результаты}
Четко видно, что ошибка прямопропорциональна шуму. Однако, более важно тут то, что на грубую сходимость системы это никак не влияет и в начале графики выглядят идентично. Разница становится заметна только при значениях ошибки уже близким к 0 -- система с большой погрешностью в заметно более широкой окрестности цели.
\begin{figure}[h!]
    \centering
    \includegraphics[width=\textwidth]{task3}
    \caption{Результат выполнения третьего задания.}
    \label{fig:fig3}
\end{figure}

\pagebreak

\subsection{Задание 4. Задача слежения для системы с астатизмом нулевого порядка.}

\subsubsection{Теория}
В этом задании будет проведено исследование слежения системы с астатизмом нулевого порядка  при различных входных воздействиях. 
\subsubsection{Результаты}
На графике представлены поведение системы при различных коэффициентах \(k\). Заметно, что при константном воздействии (рис. \ref{fig:fig4}) он уменьшает ошибку. Ее предельное значение было посчитано через предельную теорему и представлено в легенде.
\begin{figure}[h!]
    \centering
    \includegraphics[width=\textwidth]{task4_const}
    \caption{Система с астатизмом 0. Константное воздействие.}
    \label{fig:fig4}
\end{figure}

На графике (рис. \ref{fig:fig5}) представлено поведение системы при линейном воздействии. Графики расходятся -- ошибка стремится к бесконечности. 
\begin{figure}[h!]
    \centering
    \includegraphics[width=\textwidth]{task4_lin}
    \caption{Система с астатизмом 0. Линейное воздействие.}
    \label{fig:fig5}
\end{figure}

На графике (рис. \ref{fig:fig6}) представлено поведение системы при переодическом воздействии. Ошибка стремится к 0. 
\begin{figure}[h!]
    \centering
    \includegraphics[width=\textwidth]{task4_sin}
    \caption{Система с астатизмом 0. Переодическое воздействие.}
    \label{fig:fig6}
\end{figure}
\pagebreak

\subsection{Задание 5. Задача слежения для системы с астатизмом первого порядка.}

\subsubsection{Теория}
Задание аналогично предыдущему, только на этот раз ПИ регулятор, который повышает порядок астатизма.
\subsubsection{Результаты}
Сначала было проведенно влияние П коэффициента.
Заметно, что при константном воздействии (рис. \ref{fig:fig7}) его влияние уже не столь очевидно. Так же заметен вклад И части -- ошибка всех графиков сходится к 0.
\begin{figure}[h!]
    \centering
    \includegraphics[width=\textwidth]{task5_k0_const}
    \caption{Система с астатизмом 0. Константное воздействие.}
    \label{fig:fig7}
\end{figure}
При линейном воздействии (рис. \ref{fig:fig8}) ошибка никак не зависит от П коэффициента.
\begin{figure}[h!]
    \centering
    \includegraphics[width=\textwidth]{task5_k0_lin}
    \caption{Система с астатизмом 0. Линейное воздействие.}
    \label{fig:fig8}
\end{figure}

При переодическом воздействии (рис. \ref{fig:fig9}) влияние коэффициента П определить крайне тяжело.
\begin{figure}[h!]
    \centering
    \includegraphics[width=\textwidth]{task5_k0_sin}
    \caption{Система с астатизмом 0. Переодическое воздействие.}
    \label{fig:fig9}
\end{figure}


При константном воздействии (рис. \ref{fig:fig10}) влияние И очень заметно. Он ускорят время переходного процесса, но при этом вызывает перерегулирование.
\begin{figure}[h!]
    \centering
    \includegraphics[width=\textwidth]{task5_k1_const}
    \caption{Система с астатизмом 0. Константное воздействие.}
    \label{fig:fig10}
\end{figure}
При линейном воздействии (рис. \ref{fig:fig11}) ошибка обратно пропорциональна И.
\begin{figure}[h!]
    \centering
    \includegraphics[width=\textwidth]{task5_k1_lin}
    \caption{Система с астатизмом 0. Линейное воздействие.}
    \label{fig:fig11}
\end{figure}

При переодическом воздействии (рис. \ref{fig:fig12}) влияние коэффициента И определить крайне тяжело.
\begin{figure}[h!]
    \centering
    \includegraphics[width=\textwidth]{task5_k1_sin}
    \caption{Система с астатизмом 0. Переодическое воздействие.}
    \label{fig:fig12}
\end{figure}
\pagebreak
\newpage


\subsection{Задание 6. Исследование линейной системы замкнутой регулятором общего вида.}

\subsubsection{Теория}
В этом задании был протестирован принцип внутренней модели и получена управляемая система.
\subsubsection{Результаты}
Благодаря принципу замкнутой модели был синтезирован регулятор для управления системой.
\begin{figure}[h!]
    \centering
    \includegraphics[width=\textwidth]{task6}
    \caption{Результат синтеза регулятора.}
    \label{fig:fig13}
\end{figure}

\pagebreak
\pagebreak
\section{Заключение}
В этой работе было проведенно исследование следующих вопросов:
\begin{itemize}
    \item Астатизмы.
    \item Принцип внутренней модели.
    \item Идеальное и реальное дифференцирующие звенья
\end{itemize} 
\subsection{Выводы}
\begin{enumerate}
   \item Проведено моделирование вынужденного движение систем с различным ненулевым входным воздействием.
   \item На практике проверенно влияние мод на характер поведения системы.
   \item Наглядно проверенно, что свертка оригиналов равносильна перемножению образов Лапласа.
\end{enumerate}

\end{document}