\documentclass[16pt]{article}

\usepackage{report}

\usepackage[utf8]{inputenc} % allow utf-8 input
\usepackage[english, russian]{babel}
\usepackage[T1]{fontenc}    % use 8-bit T1 fonts
\usepackage[colorlinks=true, linkcolor=black, citecolor=blue, urlcolor=blue]{hyperref}       % hyperlinks
\usepackage{url}            
\usepackage{booktabs}       
\usepackage{amsfonts}       
\usepackage{nicefrac}      
\usepackage{microtype} 
\usepackage{graphicx}
\usepackage{natbib}
\usepackage{doi}
\usepackage{mathtools}
\usepackage{minted}
\usepackage{graphicx}
\graphicspath{ {./figs/} }


\setcitestyle{aysep={,}}



\title{ЛР \textnumero 2 <<Переходные процессы, свободное движение, устойчивость>>}

\author{
Студент \\
Кирилл Лалаянц\\
R33352\\
336700\\
Вариант - 6\\
\\
Преподаватель\\
Пашенко А.В. \\
\\
\\
Факультет Систем Управления и Робототехники\\
\\
ИТМО\\
}

% Uncomment to remove the date
\date{13.09.2023}

% Uncomment to override  the `A preprint' in the header
\renewcommand{\headeright}{ЛР \textnumero 2 <<Переходные процессы, свободное движение, устойчивость>>}
\renewcommand{\undertitle}{Отчет}
\renewcommand{\shorttitle}{}


\begin{document}
\maketitle
\newpage
\tableofcontents
\thispagestyle{empty}

\newpage
\setcounter{page}{1}
\section{Вводные данные}
\subsection{Цель работы}
В этой работе будет проведенно исследование следующих вопросов:
\begin{itemize}
    \item Характер свободного поведения системы в зависимости от корней ее характеристического уравнения. 
    \item Исследования границ зоны устойчивости конкретной системы. 
    \item Создание автономного генератора для воссаздания поведения системы. 
    \item Изучение фазовых портретов канонической управляемой формы.
\end{itemize} 

\subsection{Инициализация необходимых переменных в Python}
Импорт необходимых библиотек (для реализации здесь и далее используется \href{https://python-control.readthedocs.io}{Python Control Systems Library}); инициализация массива временных отметок:
\begin{minted}[linenos, fontsize=\footnotesize,numbersep=5pt, frame=lines, framesep=2mm]{python}
import matplotlib.pyplot as plt
import numpy as np
import control 
import sympy
import os
import math
\end{minted}

\newpage
\section{Выполнение работы}
\label{sec:headings}


\subsection{Задание 1. Свободное движение.}

\subsubsection{Теория}
Хотим получить ДУ системы вида:
\[\ddot{y} + a_1\dot{y} + a_0y = u\]
Cначала получим характеристическое уравнение для частного решения вида:
\[\ddot{y} + a_1\dot{y} + a_0y = 0\]
Имеем два корня -- \(\lambda_1\) и \(\lambda_2\). Тогда:
\[ (p - \lambda_1)(p-\lambda_2) = p^2 - (\lambda_1 + \lambda_2)p + \lambda_1\lambda_2 = 0,\]
умножив которое на \(y\) получаем:
\[\ddot{y} - (\lambda_1 + \lambda_2)\dot{y} + \lambda_1\lambda_2y = 0.\]

Для данного ДУ известно частичное решение:
\[ y_{0}(t) = c_1 e^{\lambda_1 t} + c_2 e^{\lambda_2 t},\]
параметры \(c_1\) и \(c_2\) которого вычисляются из начальных условий.
\\
Также подставим характеристическое уравнение в знаменатель TF \(W(p)\):
\[W(p) = \frac{1}{p^2 - (\lambda_1 + \lambda_2)p + \lambda_1\lambda_2}\]
Полученная TF соотвествует системе с желаемыми корнями характеристического уравнения.
\subsubsection{Программная реализация}
\begin{minted}[linenos, fontsize=\footnotesize,numbersep=5pt, frame=lines, framesep=2mm]{python}
def task1_output(m1, m2, initial_value, ts, plot_name, save_name):
    poly = sympy.simplify((p - m1) * (p - m2))
    coeffs = sympy.Poly(poly, p).all_coeffs()
    
    ss = control.tf2ss(control.tf(1, np.array(coeffs, dtype=np.float64)))
    ss_reachable = control.canonical_form(ss, form="reachable")[0]

    tf2_y_0_0 = control.forced_response(ss_reachable, U=0, X0=[0, 0], T=ts)
    tf2_y_1_1 = control.forced_response(ss_reachable, U=0, X0=initial_value, T=ts)
    plot_task1(tf2_y_0_0, tf2_y_1_1, initial_value, ts, plot_name, save_name)
\end{minted}

\subsubsection{Результаты}
\begin{itemize}
    \item Устойчивая и неустойчивая апереодические моды 
    

    Выбраны \(\lambda_1 = -1\) и \(\lambda_2 = 1\). Получено уравнение свободного движения вида:
    \[y_0(t)= c_1 e^{-t} + c_2 e^t\]
    Из аналитического решения видно, что система, при отличных от 0 начальных условиях, со временем стремится к бесконечноcти, следовательно -- неустойчива.
    
    
    Проверим моделированием:

    \begin{figure}[H]
        \centering
        \includegraphics[width=\textwidth]{task1_2}
        \caption{Результаты устойчивой и неустойчивой апереодических мод.}
        \label{fig:fig1}
    \end{figure}

    Действительно, все так.

    \item Нейтральная и неустойчивая апереодические моды 
    
    
    Выбраны \(\lambda_1 = 0\) и \(\lambda_2 = 1\). Получено уравнение свободного движения вида:
    \[y_0(t)= c_1 e^{-t} + c_2\]
    Из аналитического решения видно, что система, при отличных от 0 начальных условиях, со временем стремится к бесконечноcти, следовательно -- неустойчива.
    
    
    Проверим моделированием:

    \begin{figure}[H]
        \centering
        \includegraphics[width=\textwidth]{task1_4}
        \caption{Результаты нейтральной и неустойчивой апереодических мод.}
        \label{fig:fig2}
    \end{figure}

    Действительно, все так.

    \item Пара консервативных мод
    
    
    Выбраны \(\lambda_1 = i\) и \(\lambda_2 = -i\). Получено уравнение свободного движения вида:
    \[y_0(t)= c_1 cos(t) + c_2 sin(t)\]
    Из аналитического решения видно, что система, при отличных от 0 начальных условиях, устойчива по Ляпунову.
    
    
    Проверим моделированием:

    \begin{figure}[H]
        \centering
        \includegraphics[width=\textwidth]{task1_6}
        \caption{Результаты пары консервативных мод.}
        \label{fig:fig3}
    \end{figure}

    Действительно, все так.
\end{itemize}



\newpage
\subsection{Задание 2. Область устойчивости. }
\subsubsection{Теория}
Возьмем первый набор корней из первого задания \(\lambda_1 = -1\) и \(\lambda_2 = 1\).
Представим соответсвующие им TF вида:
\[W(p) = \frac{1}{Tp + 1}\]
Получаем:
\[W_1(p) = \frac{1}{T_1p + 1} = \frac{1}{-1T_1 + 1} \rightarrow T1 = 1\]
\[W_2(p) = \frac{1}{T_2p + 1} = \frac{1}{1T_2 + 1} \rightarrow T2 = -1\]

Так же по задание имеем:
\[W_3(p) = \frac{1}{p}\]
\[W_4(p) = K\]
\[W_{feedback}(p) = -1\]

Тогда получаем общую TF cистемы:
\[W(p) = \frac{W_1(p)W_2(p)W_3(p)W_4(p)}{1 + W_1(p)W_2(p)W_3(p)W_4(p)} = \frac{K}{T_1T_2p^3 + (T_1+T_2)p^2+p+K}\]

Для анализа устойчивости воспользуемся следствием из критерия Гурвица для однородного ДУ третьего порядка:

\[
    \begin{cases}
        \frac{K}{T_1T_2} > 0 \\
        \frac{T_1+T_2}{T_1T_2} > 0\\ 
        \frac{K}{T_1T_2} > 0\\ 
        \frac{T_1+T_2}{(T_1T_2)^2} > \frac{K}{T_1T_2}
    \end{cases},
\]

\begin{itemize}
    \item Зафиксируем значение \(T_2 = -1\)
    
    Тогда система принимает вид: 
    \[
    \begin{cases}
        \frac{K}{T_1(-1)} > 0 \\
        \frac{T_1+(-1)}{T_1(-1)} > 0\\ 
        \frac{K}{T_1(-1)} > 0\\ 
        \frac{T_1+(-1)}{(T_1(-1))^2} > \frac{K}{T_1(-1)}
    \end{cases},
    \]
    решений у неё нет. Следовательно, система не может быть устойчивой с таким параметром \(T_2\).

    \item Зафиксируем значение \(T_1 = 1\)
    
    Тогда система принимает вид: 
    \[
    \begin{cases}
        \frac{K}{T_2} > 0 \\
        \frac{T_2+1}{T_2} > 0\\ 
        \frac{K}{T_2} > 0\\ 
        \frac{T_2+1}{T_2^2} > \frac{K}{T_2}
    \end{cases},
    \]
    у нее есть решения, представленные на графике ниже.
    \begin{figure}[H]
        \centering
        \includegraphics[width=\textwidth]{task2_T2_K}
        \caption{Зависимость устойчивости от K и \(T_2\) при \(T_1 = 1\) .}
        \label{fig:fig4}
    \end{figure}
\end{itemize}
\newpage
\subsubsection{Результаты}
\begin{figure}[h]
	\centering
	\includegraphics[width=\textwidth]{task2_conservative}
	\caption{Устойчивые по Ляпунову параметры.}
	\label{fig:fig5}
\end{figure}
\begin{figure}[H]
	\centering
	\includegraphics[width=\textwidth]{task2_stable}
	\caption{Устойчивые ассимптотически параметры.}
	\label{fig:fig6}
\end{figure}
\begin{figure}[H]
	\centering
	\includegraphics[width=\textwidth]{task2_not_stable}
	\caption{Неустойчивые параметры.}
	\label{fig:fig7}
\end{figure}
По приведенным выше графикам видно, что полученная зона устойчивости соответсвует действительности.
\newpage
\subsection{Задание 3. Автономный генератор.}
\subsubsection{Теория}
Необходимо найти матрицы \(A\) и \(C\) такие, чтобы выход системы совпадал с функцией:
\[y(t) = cos(5t) + e^t + e^{-5t}\]

Из функции ясно видно, что модами будут:
\(\lambda_1 = 1; \lambda_2 = -5; \lambda_3 = 5i; \lambda_4 = -5i.\)

Запишем в вещественной жордановой форме:
\[
    A = \begin{bmatrix}
    1 & 0 & 0 & 0 \\
    0 & -5 & 0 & 0 \\
    0 & 0 & 0 & 5 \\
    0 & 0 & -5 & 0 \\
    \end{bmatrix},
\]

Известно, что:
\[ x_0(t) = e^{At}x(0)  \]
\[ y_0(t) = Ce^{At}x(0) \]

Вычислим матричную экспоненту: 
\[
    e^{At} = \begin{bmatrix}
        1 & 0 & 0 & 0 \\
        0 & 1 & 0 & 0 \\
        0 & 0 & 1 & 0 \\
        0 & 0 & 0 & 1 \\
        \end{bmatrix}
        +
        \begin{bmatrix}
            t & 0 & 0 & 0 \\
            0 & -5t & 0 & 0 \\
            0 & 0 & 0 & 5t \\
            0 & 0 & -5t & 0 \\
        \end{bmatrix}
        +
        \begin{bmatrix}
            t^2/2 & 0 & 0 & 0 \\
            0 & 25t^2/2 & 0 & 0 \\
            0 & 0 & -25t^2/2 & 0 \\
            0 & 0 & 0 & -25t^2/2 \\
        \end{bmatrix}
        +
        ...;
\]
После чего заметно, что результатами сложения отдельных ячеек являются ряды Тейлора, откуда получаем:
\[
    e^{At} = \begin{bmatrix}
        e^t & 0 & 0 & 0 \\
        0 & e^{-5t} & 0 & 0 \\
        0 & 0 & cos(5t) & sin(5t) \\
        0 & 0 & -sin(5t) & cos(5t) \\
        \end{bmatrix}
\]
Зададим вектор начальных условий и матрицу \(C\):
\[
    x(0) = \begin{bmatrix}
        a_1 \\ 
        a_2 \\
        a_3 \\
        a_4
        \end{bmatrix},
    C = \begin{bmatrix}
        c_1 \\ 
        c_2 \\
        c_3 \\
        c_4
        \end{bmatrix};
\]
тогда:
\[
    y_0(t) = Ce^{At}x(0) = a_1c_1e^t + a_2c_2e^{-5t} + (a_3c_3 + a_4c_4)cos(5t) + (-a_3c_4+a_4c_3)sin(5t)
\]
Отсюда, сопоставив с исходной функцией, получаем систему уравнений:
\[
    \begin{cases}
        a_1c_1 = 1\\
        a_2c_2 = 1\\ 
        a_3c_3 + a_4c_4 = 1\\ 
        -a_3c_4+a_4c_3 = 0
    \end{cases},
    \]
Мы можем самостоятельно выбрать любой вектор начальных условий, что сводит систему уравнений к системе из 4 уравнений с 4 неизвестными, после чего получаем:
\[
    x(0) = \begin{bmatrix}
        1 \\ 
        1 \\
        1 \\
        1
        \end{bmatrix},
    C = \begin{bmatrix}
        1 \\ 
        1 \\
        0.5 \\
        0.5
        \end{bmatrix},
    A = \begin{bmatrix}
        1 & 0 & 0 & 0 \\
        0 & -5 & 0 & 0 \\
        0 & 0 & 0 & 5 \\
        0 & 0 & -5 & 0 \\
        \end{bmatrix};
\]
\subsubsection{Программная реализация}
\begin{minted}[linenos, fontsize=\footnotesize,numbersep=5pt, frame=lines, framesep=2mm]{python}
task3_ss = control.ss(task3_A, np.zeros((4, 1)), task3_C, 0)
ss_response = control.forced_response(task3_ss, ts, X0=task3_initial)
\end{minted}
\newpage

\subsubsection{Результаты}
\begin{figure}[H]
	\centering
	\includegraphics[width=\textwidth]{task3}
	\caption{Результаты третьего задания.}
	\label{fig:fig8}
\end{figure}
По графику видно, что генератор совпал с исходной функцией. В легедне указана получившаяся квадратичная ошибка, которая очень мала и просто является погрешностью вычисленний Python3.


\newpage
\subsection{Задание 4. (Необязательное) Изучение канонической управляемой формы:
фазовые портреты.}
\subsubsection{Теория}

Рассмотрим фазовый портрет системы 3 из первого задания.
\[\ddot{y} + y = 0\]

Её TF имеет вид:
\[W(p) = \frac{1}{p^2 + 1}\]

В форме SS имеет вид:
\[
    A = \begin{bmatrix}
        0 & 1  \\
        -1 & 0 \\
        \end{bmatrix}, 
    C = \begin{bmatrix}
        1 & 0  \\
        \end{bmatrix},
    B = D = 0; 
\]

\subsubsection{Программная реализация}
\begin{minted}[linenos, fontsize=\footnotesize,numbersep=5pt, frame=lines, framesep=2mm]{python}
def task4_output(m1, m2, initial_values, ts):
    plot_name = f'Моды: {m1, m2}'
    save_name = f'task4_{m1}_{m2}.jpg'
    poly = sympy.simplify((p - m1) * (p - m2))
    coeffs = sympy.Poly(poly, p).all_coeffs()
    tf_y = control.tf(1, np.array(coeffs, dtype=np.float64))
    tf_dy = tf_y * control.tf([1, 0], [1]) 
    ss = control.tf2ss(tf_dy)
    ss_reachable = control.canonical_form(ss, form="reachable")[0]
    
    ss = control.tf2ss(tf_y)
    ss_reachable = control.canonical_form(ss, form="reachable")[0]

    responses = []
    for initial_value in initial_values:
        tf_y_response = control.forced_response(tf_y, U=0, X0=initial_value, T=ts)
        tf_dy_response = control.forced_response(tf_dy, U=0, X0=initial_value, T=ts)
        ss_response = control.forced_response(ss_reachable, U=0, X0=initial_value, T=ts)
        responses.append([initial_value, tf_y_response, tf_dy_response, ss_response])
    
    task4_plot(responses, plot_name, save_name)
\end{minted}

\newpage
\subsubsection{Результаты}
\begin{figure}[H]
	\centering
	\includegraphics[width=\textwidth]{task4_-I_I}
	\caption{Результаты четвертого задания.}
	\label{fig:fig9}
\end{figure}
Полученные графики совпали, что в целом ожидаемо, потому что вектор состояния для системы второго порядка из себя и представляет \(\dot{y}\) и \(y\).

\newpage

\section{Заключение}
В этой работе было проведенно исследование следующих вопросов:
\begin{itemize}
    \item Характер свободного поведения системы в зависимости от корней ее характеристического уравнения. 
    \item Исследования границ зоны устойчивости конкретной системы. 
    \item Создание автономного генератора для воссаздания поведения системы. 
    \item Изучение фазовых портретов канонической управляемой формы.
\end{itemize} 
\subsection{Выводы}
\begin{enumerate}
   \item Проведено моделирование свободное движение систем с различным ненулевым входным воздействием.
   \item На практике проверенно влияние мод на характер поведения системы.
   \item На практике проверенна верность работы критерия Гурвица.
   \item Создан генератор для воссоздания поведения функции.
   \item Рассмотренно фазовые диаграмы системы. Проверенно, что вектор состояния соответсвует \(y\) и его производным. 
\end{enumerate}

\end{document}